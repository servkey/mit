\documentclass{article}
\usepackage[spanish]{babel}
\usepackage[utf8,latin1]{inputenc}
\usepackage{graphicx}


\graphicspath{{images/}}
\title{Trabajando con figuras}
\author{Luis G. Montan�-Jim�nez}

\begin{document}
\maketitle
%Cambiar cuadro por tabla
\renewcommand{\tablename}{Tabla}
\renewcommand{\listtablename}{�ndice de tablas}
%Generar �ndice de tablas
\listoftables
%Generar �ndice de figuras
\listoffigures
\newpage


La Figura~\ref{fig:world} representa el planeta tierra. Por otro lado, el planeta marte est� representado por la Figura~\ref{fig:mars}. La Tabla~\ref{tab:planetas} resume caracter�sticas de ambos planetas.

	\begin{figure}[h]
		\centerline{\includegraphics[height=2cm]{world.png}}
		\caption{Planeta tierra}
		\label{fig:world}
	\end{figure}

	\begin{figure}[h]
		\centerline{\includegraphics[width=2cm]{mars.png}}
		\caption{Planeta marte}
		\label{fig:mars}
	\end{figure}
	
	\begin{table}[h]
		\caption{Caracter�sticas relevantes de los planetas}
		\centerline{
				\begin{tabular}{l|cc}
						\hline
						Caracter�stica	&	Tierra							&	Marte 									\\	\hline
						Masa						&	5,9736x$10^{24}$kg			&	6,4185x$10^{23}$kg 			\\	\hline		
				\end{tabular}
		}
		\label{tab:planetas}
	\end{table}	
	
\end{document}