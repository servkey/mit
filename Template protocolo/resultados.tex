\section{Cronograma}

\section{Presupuesto}

\section{Difusi�n}

\section{Resultados}

\subsection{An�lisis de resultados}
Posterior al ejercicio de la experimentaci�n, se llevar� a cabo el an�lisis de resultados utilizando herramientas tecnol�gicas para detecci�n de comportamiento del usuario, cuestionarios de usabilidad y bit�coras con el registro de actividades individuales y grupales que podr�an ser generadas por el prototipo de sistema colaborativo que se defina para la experimentaci�n.

\subsection{Metas propuestas con la tesis}
A trav�s de la ejecuci�n de las fases del proyecto (preparaci�n, modelado, construcci�n y validaci�n experimental), se pretende elaborar un modelo de visualizaci�n de desempe�o de equipos en sistemas colaborativos y demostrar que la visualizaci�n de indicadores de desempe�o apoya a la comunicaci�n, coordinaci�n e interacci�n de los equipos de trabajo en sistemas colaborativos.
El modelo propuesto permitir� la representaci�n de los indicadores de desempe�o que sean necesarios seg�n las caracter�sticas de los diversos sistemas colaborativos, tambi�n se obtendr� informaci�n sobre la granularidad que pueda permitir una visualizaci�n de los datos durante la ejecuci�n de las actividades de un equipo dentro de un sistema colaborativo y se definir� si se requiere una figura adicional para el \textit{couching} de los equipos de trabajo.