\chapter{Análisis de Ambientes Ubicuos y las TICs}
En las últimas décadas se ha demostrado que el uso de las Tecnologías de Información y comunicación ha beneficiado el crecimiento y desarrollo de la sociedad, brindando apoyo para la realización de sus actividades y apoyo en la toma de decisiones. Así como las TICs han ayudado a la evolución de la sociedad, el área computacional también ha ido en crecimiento, recientemente empresas e investigadores se han enfocado en estudiar y desarrollar herramientas que apoyen el trabajo social en conjunto y aprovechando la proliferación tecnológica \citep{Duran2017}.

En la actualidad, se ha evidenciado que instituciones, tanto públicas como privadas, han evolucionado su forma de trabajar, exigiendo estar actualizadas y facilitando funciones de comunicación, entrega y disponibilidad personalizada de recursos físicos y de información, con el objetivo de mejorar la forma de elaboración, impartición y desarrollo de actividades escolares. Por esto, la detección y seguimiento del uso de herramientas de software o hardware eventualmente podrían mejorar el proceso de cooperación de los profesores con sus estudiantes, para lograr formas eficientes de realizar una o varias actividades individuales o colaborativas, y teniendo como resultado mayor producción y calidad del trabajo.

Es por lo descrito anteriormente, que las aplicaciones actuales en su gran mayoría carecen de un seguimiento para el usuario \cite{MontaneLuis;Toledo2015}, no obstante, han surgido guías y metodologías que proponen el diseño sistemas que permitan que un grupo de personas realicen su trabajo de forma simple y natural, permitiendo, además, la interacción entre usuario y el sistema de forma que se puedan efectuar tareas individuales y cooperativas \citep{LuisaM.LuisJ.VisitacionM.&Ramon2007}.
\section{Uso del Internet de las Cosas y Ambientes Asistidos}
Hoy en día, el Internet de las Cosas ha cambiado la forma en que interactuamos con un sistema de información. Con el sistema de información dominante que enfrenta varias complejidades de los clientes finales, el modo de desarrollo basado en la asociación semántica y la toma de conciencia del contexto es posible proporcionar un servicio personalizado para cada usuario \citep{Liu2018, Virtanen2017}. En algunos trabajos del área, la inteligencia ambiental basada en el conocimiento del contexto busca predecir la intención de los usuarios a partir de los contextos obtenidos por las fuentes de datos. Al proponer la predicción a los servicios logísticos, sería posible proporcionar un servicio personalizado para mantener a los clientes satisfechos. Una cuestión clave en los servicios centrados en el usuario es cómo detectar una determinada situación con un usuario específico, y elegir un determinado servicio que satisfaga los requisitos de los usuarios de la mejor manera, proporcionando a su vez soporte en tiempo real para apoyar la toma de decisión \citep{Sezer2017}. De esta manera, se piensa que el cumplimiento de la inteligencia ambiental no se puede separar del soporte tecnológico cuando se trata de comportamiento inteligente.

Los ambientes inteligentes y ambientes asistidos tradicionalmente son abordados desde el área de Internet de las Cosas e Inteligencia Ambiental, buscando soluciones para generar espacios donde la elaboración de cierto tipo de actividades sea eficiente, y apoyando la construcción de ambientes físicos de siguiente generación. Por esto, en este documento se plantea y describe un escenario en un laboratorio de cómputo para ejemplificar situaciones que ocurren hoy en día en actividades académicas \citep{Virtanen2017}, con la finalidad de ejemplificar la integración de un ambiente asistido automatizado y teniendo en cuenta variables y comportamientos emergentes en actividades escolares.

Con el análisis realizado en la literatura actual, se han podido identificar variables que eventualmente influyen en el seguimiento y realización de actividades por parte de los estudiantes. Estas variables serían tomadas en cuenta como datos contextuales que serían adquiridos desde fuentes de datos heterogéneas, como, por ejemplo, los mismos equipos de cómputo que utilizan los estudiantes para trabajar en sus actividades. De las variables involucradas durante la actividad académica se encuentran, tiempo transcurrido a partir del inicio de la actividad, las aplicaciones o programas informáticos activos y ejecutados dentro del sistema operativo, páginas Web abiertas y activas en un navegador, tiempo de actividad e inactividad del estudiante dentro del Sistema Operativo, carga de procesamiento del CPU y memoria RAM utilizada. Por otro lado, entre las variables involucradas con el aspecto ambiental se encuentran, nivel de luz, ruido y temperatura. Estos son ejemplos de variables que es importante considerarlas, ya que pueden afectar de manera negativa o positiva el rendimiento académico de un estudiante dentro de un aula.

La revisión de trabajos relacionados presentados en esta sección se realizó con la finalidad de proponer una solución que apoye las problemáticas identificadas. Con la evidencia encontrada en esta revisión, se pudo identificar que el utilizar un ambiente asistido automatizado podría apoyar el desarrollo de las actividades realizadas en aulas de cómputo.


\section{Gestion de datos contextuales en ambientes físicos}
Para fines de este documento, un aula de cómputo es un espacio de colaboración entre estudiantes y profesores utilizado para alcanzar un objetivo común, el cual puede ser aprender a usar una herramienta de ofimática o programar un algoritmo específico. Un área que se encarga de estudiar las actividades colaborativas y este tipo de escenarios es el CSCW (Computer-Supported Cooperative Work), el cual aborda temáticas y el conocimiento que tienen los individuos sobre sí mismos y sobre el ambiente que los rodea, y en el caso de trabajo colaborativo, el rol que desempeñan en su grupo y el estado de los demás integrantes. Desde este enfoque, es importante que los individuos sepan lo que están haciendo los demás ya que pueden usar ese conocimiento para anticipar las acciones de los otros, y ayudarlos con sus tareas \citep{Gutwin1996}. Dentro del CSCW, el término awareness (o consciencia) fue introducido por primera vez por \citep{Dourish1992a} definiéndolo como el entendimiento de las actividades de otros, lo cual proporciona un contexto para tu propia actividad, y donde este contexto es usado para asegurar que las contribuciones individuales sean relevantes a las actividades del grupo, y para evaluar las acciones individuales con respecto a los objetivos del grupo y sus progresos. Por lo que el dato contextual o el contexto es información que contribuye en el proceso de awareness.

Tradicionalmente, los datos de contexto son relacionados con la localización, identidad, y estado de las personas, grupos y objetos virtuales y físicos, según \citep{Souza2013}. Por lo que el contexto puede ser visto como un conjunto de condiciones e influencias en una situación relevante y que la hacen única y comprensible, esta situación puede referirse a una persona, grupo de personas, objeto físico, entidad computacional, entre otras. El concepto de modelos mentales tiene una relación cercana a la consciencia contextual y situacional \citep{Aehnelt2012}, al momento de modelar contexto, es necesario distinguir entre los diferentes tipos de información contextual \citet{Hoyos2013}, el contexto de las actividades pueden ir desde un editor de documentos colaborativo, hasta un videojuego de un género en particular, surgiendo de esta manera la necesidad de utilizar una taxonomía con un alto nivel de abstracción que soporte la diversidad de contexto con los dominios aplicativos con los que se trabaja. \citet{Eynard2013} proponen un modelo situacional para mejorar la consciencia centrada en los conceptos de situación, interacción y rol, en el que hacen un recuento de los conceptos clave a considerar en un modelo, entre ellos se encuentran: i) elemento contextual, ii) tarea, iii), recursos, iv) interacción y v) rol. Estos elementos podrían ser contemplados para generar una infraestructura tecnológica que apoye en la generación de ambientes asistidos automatizados.

Por lo tanto, el conocimiento contextual junto con el awareness describe una situación y la forma en la que se usan los elementos en un grupo de trabajo, incluyendo los eventos que son manejados por el grupo \citep{Brezillon2004}. En una revisión de trabajos realizados anteriormente, \citet{Dey2001} propone una definición con un enfoque computacional, refiriéndose a contexto como cualquier tipo de información que se puede usar para caracterizar la situación de entidades (se entiende por entidad una persona, lugar u objeto) que es considerada relevante para la interacción entre un usuario y una aplicación incluyendo al usuario y la aplicación. Existen problemas en la actualidad inherentes a consciencia contextual, por ejemplo, la definición del contexto, la cual puede abarcar todos los posibles parámetros que identifican una situación, las aplicaciones y marcos de trabajo están limitados a definir los de su propio dominio. Otro problema importante es que aún las arquitecturas carecen de estar completamente desarrolladas, están elaboradas para cumplir con tareas específicas, y requieren del diseño estándares para definir arquitecturas o herramientas tecnológicas. Por último, otro problema de interés para este documento es la interpretación del contexto y las adaptaciones del comportamiento del servicio \citep{Mahmud2007}.

Considerando lo anterior, es que las soluciones existentes en el ámbito de consciencia contextual deben dar soporte a la conciencia de equipo, también conocida como awareness \citep{Dourish1992a}. Es decir, incluir información sobre quién está utilizando el sistema, dónde están trabajando y qué están haciendo. También es importante que las acciones de un usuario se muestren a los otros usuarios que colaboran en la misma tarea. Por ejemplo, observar a otra persona navegar a través de los elementos de un menú suministra indicios sobre lo que está haciendo o quiere hacer. Estos requisitos resaltan la importancia de proporcionar el soporte adecuado a la construcción de interfaces para aplicaciones colaborativas que se adapten a un ambiente de trabajo.
