\section{Marco te�rico}


\section{Estado del arte}

Los Sistemas Colaborativos (SC) o Groupware derivan del campo de estudio Trabajo Colaborativo Asistido por Computadora (CSCW, por sus siglas en ingl�s), este se enfoca en el estudio de grupos de trabajo y busca descubrir c�mo el c�mputo puede apoyarlos. Como �rea de estudio interdisciplinaria, CSCW involucra ciencias sociales (p.ej.psicolog�a, sociolog�a, teor�a organizacional, antropolog�a, entre otras) y ciencias de la computaci�n (p.ej. inteligencia artificial, sistemas distribuidos, dise�o de interfaz de usuario y usabilidad) \cite{Mills2003}. En el �rea de c�mputo destacan los SC, son sistemas basados en computadora que soportan grupos de gente comprometidos en una tarea com�n (o meta) y que proveen una interfaz para un ambiente compartido \cite{Clarence1991}.


Los SC consideran los aspectos sociales, pero est�n m�s enfocados en los computacionales: el espacio, tiempo, cantidad de usuarios, entre otros, son interpretados como dimensiones que heredan de los CSCW. Estas dimensiones deben tomarse en cuenta para el dise�o de los SC en aspectos de comunicaci�n, coordinaci�n e interacci�n de usuarios, entre otros, que var�an sus caracter�sticas seg�n la dimensi�n que sea m�s relevante en el SC (p.ej. un SC con comunicaci�n a distancia). 



